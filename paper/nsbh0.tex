% ****** Start of file apssamp.tex ******
%
%   This file is part of the APS files in the REVTeX 4.2 distribution.
%   Version 4.2a of REVTeX, December 2014
%
%   Copyright (c) 2014 The American Physical Society.
%
%   See the REVTeX 4 README file for restrictions and more information.
%
% TeX'ing this file requires that you have AMS-LaTeX 2.0 installed
% as well as the rest of the prerequisites for REVTeX 4.2
%
% See the REVTeX 4 README file
% It also requires running BibTeX. The commands are as follows:
%
%  1)  latex apssamp.tex
%  2)  bibtex apssamp
%  3)  latex apssamp.tex
%  4)  latex apssamp.tex
%
\documentclass[%
 reprint,
 superscriptaddress,
%groupedaddress,
%unsortedaddress,
%runinaddress,
%frontmatterverbose, 
%preprint,
%preprintnumbers,
 nofootinbib,
%nobibnotes,
%bibnotes,
 amsmath,amssymb,
 aps,
%pra,
%prb,
%rmp,
%prstab,
%prstper,
%floatfix,
]{revtex4-2}

\usepackage{graphicx}% Include figure files
\usepackage{dcolumn}% Align table columns on decimal point
\usepackage{bm}% bold math
\usepackage{hyperref}% add hypertext capabilities

%\usepackage[showframe,%Uncomment any one of the following lines to test 
%%scale=0.7, marginratio={1:1, 2:3}, ignoreall,% default settings
%%text={7in,10in},centering,
%%margin=1.5in,
%%total={6.5in,8.75in}, top=1.2in, left=0.9in, includefoot,
%%height=10in,a5paper,hmargin={3cm,0.8in},
%]{geometry}

% % % % % % % % % % % % % % % % % % % % % % % % % % % % % %

% journals
\newcommand{\mnras}{MNRAS}
\newcommand{\aap}{A\&A}
\newcommand{\aaps}{A\&AS}
\newcommand{\apjs}{ApJS}
\newcommand{\apjl}{ApJL}
\newcommand{\araa}{ARA\&A}
\newcommand{\jhep}{JHEP}
\newcommand{\aj}{AJ}
\newcommand{\jcap}{JCAP}
\newcommand{\pasp}{PASP}
\newcommand{\pnas}{PNAS}
\newcommand{\epjc}{EPJC}

% maths
%\newcommand{\dataset}{\vect{d}}
\newcommand{\msun}{M_\odot}
\newcommand{\hubble}{\ensuremath{H_0}}
\newcommand{\hubbleest}{\ensuremath{\hat{H}_0}}
\newcommand{\decel}{\ensuremath{q_0}}
\newcommand{\jerk}{\ensuremath{j_0}}
\newcommand{\dl}{\ensuremath{D}}
\newcommand{\vobs}{\ensuremath{\hat{v}}}
\newcommand{\vobss}{\ensuremath{\hat{\vect{v}}}}
\newcommand{\zobs}{\ensuremath{\hat{z}}}
\newcommand{\zmax}{\ensuremath{z_{\rm max}}}
\newcommand{\prob}{\ensuremath{{\rm P}}}
\newcommand{\normal}{{\rm{N}}}
\newcommand{\sels}{\vect{S}}
\newcommand{\inc}{\iota}
\newcommand{\nexp}{\bar{N}}
\newcommand{\abh}{a_{\rm BH}}
\newcommand{\ans}{a_{\rm NS}}
\newcommand{\mbh}{m_{\rm BH}}
\newcommand{\mns}{m_{\rm NS}}
\newcommand{\mchirp}{{\cal{M}}}
\newcommand{\strain}{h}
\newcommand{\strainobs}{\hat{h}}
\newcommand{\sigmav}{\sigma_{||}}
\newcommand{\gwpoppar}{\beta}
\newcommand{\uniform}{{\rm U}}
\newcommand{\tobs}{t_{\rm obs}}
\newcommand{\fobs}{f_{\rm obs}}
\newcommand{\rate}{\Gamma}
\newcommand{\step}{\Theta}
\newcommand{\snr}{\rho}
\newcommand{\snrmin}{\rho_*}
\newcommand{\mejmin}{m_{\rm ej}^*}
\newcommand{\dgw}{\hat{\bm{x}}}
\newcommand{\kmsmpc}{\ensuremath{{\rm km\,s^{-1}\,Mpc^{-1}}}}
\newcommand{\kms}{\ensuremath{{\rm km\,s^{-1}}}}
\newcommand{\mpc}{\ensuremath{{\rm Mpc}}}
\newcommand{\gpc}{\ensuremath{{\rm Gpc}}}
\newcommand{\yr}{\ensuremath{{\rm yr}}}
\newcommand{\yrgpc}{\ensuremath{{\rm yr^{-1}\,Gpc^{-3}}}}
\newcommand{\planck}{{\it Planck}}
\newcommand{\plancks}{{\it Planck{\rm 's}}}
\newcommand{\lcdm}{$\Lambda$CDM}

% % % % % % % % % % % % % % % % % % % % % % % % % % % % % %

\begin{document}

%\preprint{APS/123-QED}

\title{Prospects for Measuring the Hubble Constant with Neutron-Star-Black-Hole Mergers in the A+ Era}

\author{Stephen M. Feeney}
\affiliation{Department of Physics \& Astronomy, University College London, Gower Street, London WC1E 6BT, UK}
\author{Hiranya V. Peiris}
\affiliation{Department of Physics \& Astronomy, University College London, Gower Street, London WC1E 6BT, UK}
\affiliation{Oskar Klein Centre for Cosmoparticle Physics, Department of Physics,
Stockholm University, AlbaNova, Stockholm SE-106 91, Sweden}
\author{Samaya M. Nissanke}
\affiliation{GRAPPA, Anton Pannekoek Institute for Astronomy and Institute of High-Energy Physics, University of Amsterdam, Science Park 904, 1098 XH Amsterdam, The Netherlands}
\affiliation{Nikhef, Science Park 105, 1098 XG Amsterdam, The Netherlands}
\author{Daniel J. Mortlock}
\affiliation{Astrophysics Group, Imperial College London, Blackett Laboratory, Prince Consort Road, London SW7 2AZ, UK}
\affiliation{Department of Mathematics, Imperial College London, London SW7 2AZ, UK}
\affiliation{Department of Astronomy, Stockholm University, AlbaNova, SE-10691 Stockholm, Sweden}

\date{\today}% It is always \today, today,
             %  but any date may be explicitly specified

% % % % % % % % % % % % % % % % % % % % % % % % % % % % % %

\begin{abstract}
%The current controversy over the expansion rate of the Universe, \hubble, requires a precise and accurate local verification for resolution. Combined gravitational-wave (GW) and electromagnetic (EM) observations of nearby compact-object mergers are a particularly promising prospect, yielding \hubble\ estimates that depend on general relativity only. The utility of binary neutron star (BNS) mergers has already been proven by GW170817/GRB170817A, but the potential for as-yet undiscovered neutron-star-black-hole (NSBH) mergers to contribute is just beginning to be explored. Using idealized, fixed-signal-to-noise simulations at indicative parameter values, Vitale \& Chen~\cite{Vitale_Chen:2018} demonstrated that catalogues of GW-selected NSBH observations can constrain \hubble\ as well as (or better than) BNSs, depending on the relative merger rates and black hole spins. Here, we perform the first end-to-end simulations of NSBH samples, using fully specified parent populations, a complete noise treatment, and combined GW and EM selection, to determine the \hubble\ constraints realistic samples will achieve by the ``A+'' era of the mid-2020s. Using two waveforms to account for the uncertainty in the modeling of NSBH signals, we show that NSBH samples will yield unbiased \hubble\ estimates with up to 1.6 to 1.0 \kmsmpc\ precision, dependent on whether these systems have non-precessing or precessing spins. This level of precision is ideal for resolving the \hubble\ tension. The degree to which the spins precess will be critical: 
%%, accounting for an effective doubling in the sample size compared to non-precessing spins.
%thanks to their significantly reduced distance-inclination degeneracies, we find precessing-spin samples to be equivalent to non-precessing samples of twice the size.
Gravitational wave (GW) and electromagnetic (EM) observations of neutron-star-black-hole (NSBH) mergers have the potential to provide the precise, accurate and direct verification of the Hubble constant (\hubble) cosmology urgently needs. We perform the first end-to-end analysis of simulated NSBHs, incorporating both GW and EM selection, to determine that NSBHs should achieve unbiased 1-1.6 km/s/Mpc precision estimates by the mid-2020s: ideal for resolving the \hubble\ tension. The degree to which the systems' spins precess is critical: precessing-spin samples are twice as constraining as non-precessing. {\bf [600/600 chars]}
\end{abstract}

% % % % % % % % % % % % % % % % % % % % % % % % % % % % % %

%\keywords{Suggested keywords}%Use showkeys class option if keyword
                              %display desired
\maketitle

% % % % % % % % % % % % % % % % % % % % % % % % % % % % % %

\section{Introduction} \label{sec:intro}

%PRL maximum length: 3750 words; abstract should be under 600 characters.

The current expansion rate of the Universe -- the Hubble constant, or $H_0$ -- is at the center of a serious cosmological controversy. Measured directly in the local Universe by the SH0ES team's Cepheid-supernova distance ladder, it is found to be blah. Estimated in a cosmological-model-dependent fashion from the Planck satellite's observations of the cosmic microwave background's temperature and polarization anisotropies, it is found to be wah. This is a big ol' discrepancy.

There are two potential explanations here. The most exciting of the two derives from the model-dependence of the CMB constraint: could this discrepancy be due to novel physics missing from the $\Lambda$CDM model used in the Planck analysis? This has, understandably, let to a great amount of activity among theorists [cites], though consensus on a compelling theoretical explanation is yet to be reached. Arguably the strongest (least weak?) evidence points at modifications to the expansion history in the decade of scale factor preceding recombination [EDE, Knox\& Millea] (though see [hill etc.] for recent contradictory evidence).

The second, more prosaic explanation, is that there are undiagnosed analysis issues in the CMB or distance ladder data. There are two approaches here, both of which have been heavily undertaken by the field: search for undiagnosed systematic errors in the constituent datasets, and harness complementary independent datasets to verify the conclusions of the low- and high-redshift universes. On the systematics side, there have been loads of investigations into the Cepheid-SN distance ladder [Cardona, Follin, Feeney, Rigault and responses] and the CMB [Spergel, more?], none of which have found incontrovertible evidence warranting a change of conclusions. On the new data side of things, the high-z Universe is pretty much sorted: the inverse distance ladder agrees with the CMB very well, which is particularly convincing given entirely CMB-independent formulations. Low-z is a little less clear? Various distance ladders (MIRAs, tRGB controversy) and strong lensing (though again, controversy) generally back up the Cepheids. Clear that a completely independent measurement with same precision would be enormously helpful.

Here's where GWs come in. Combining GW detections, which yield distance estimates, with EM counterparts, it's possible to estimate $H_0$ assuming general relativity alone [Schutz, Holz, Samaya]. With the advent of detections from LIGO and Virgo, starting to do this, both directly in the one case where the counterpart is known (LVC) and statistically in the case where it is not (many). Prime candidate here is BNS, as counterparts expected and things much more precise [us x 2, chen]. BBH can also be useful (see recent papers on blah, but note association is not firm, and model-dependent). What hasn't been researched so much is NSBH. Discuss [Chen Vitale], and motivate what we're doing here. Full ranges of parameters. Different waveforms. EM selection. This is the focus of this paper. [Vasylyev, Alex Filippenko dark measurement?]

% % % % % % % % % % % % % % % % % % % % % % % % % % % % % %

\section{Simulations} \label{sec:sims}

In this work we simulate the results of a circa-2025 GW detector network, consisting of LIGO A+, Virgo AdV+, KAGRA and LIGO India~\cite{Abbott_etal:2013} {\bf [and Sukanta Bose private comm / LIGO-T2000158]} observing for $\tobs = 5 \yr$ with a 50\% duty cycle, $\fobs$. We assume a constant rest-frame  NSBH merger rate $\Gamma = 610$ \yrgpc\ (matching the 90\% upper limit of Ref.~\cite{Ligo:2018}), and ground truth cosmological parameters matching Ref.~\cite{Planck_VI:2018}, with $\hubble=67.36\,\kmsmpc$ and $\decel=-0.527$.

%H1+ LVC_AplusDesign_PSD.txt https://dcc.ligo.org/LIGO-T2000012/public
%L1+ LVC_AplusDesign_PSD.txt https://dcc.ligo.org/LIGO-T2000012/public
%V1+ LVC_avirgo_O5high_NEW_PSD.txt https://dcc.ligo.org/LIGO-T2000012/public
%K1+ LVC_kagra_128Mpc_PSD.txt https://arxiv.org/pdf/1102.5421.pdf
%A1 LVC_AplusDesign_PSD.txt

Each simulation starts by drawing the total number of mergers from a Poisson distribution with mean $\lambda = \fobs \, \tobs \, V \, \Gamma $, where $V$ is the redshifted volume
\begin{equation}
V = \int_0^{\zmax} \frac{{\rm d}V}{{\rm d}z} \frac{{\rm d}z}{1+z}
\end{equation}
and the co-moving volume element is
\begin{align}
\frac{{\rm d}V}{{\rm d}z} \simeq & 4\pi \frac{c^3 z^2}{\hubble^3} \\
& \left[1 - 2 \left(1 + \decel \right) z + \frac{5}{12} \left(7 + 14 \decel - 2 \jerk + 9 \decel^2 \right) z^2 \right] \nonumber
\end{align}
to third order in expansion parameters (hereafter, we assume the jerk $j_0 = 1$). The upper limit to the integral, $\zmax$, must be chosen such that there is negligible probability of a merger at $\zmax$ being detected: we find that $\zmax=0.44$ suffices for our setting. For our fiducial parameter set, $\lambda = 25160$; the particular Poisson draw we obtain yields a total of 25241 mergers.

For each merger, we draw a cosmological redshift from the distribution
\begin{equation}
\prob(z) \propto \step(z) \step(\zmax-z) \frac{1}{1+z} \frac{{\rm d}V}{{\rm d}z}
\end{equation}
(where $\step$ denotes the Heaviside step function), along with an isotropically distributed angular sky position, inclination angle, phase and polarization angle. We draw uniformly distributed BH masses from $\prob(\mbh) = \uniform(2.5\msun, 40\msun)$, taking the upper limit from low-metallicity\footnote{Using solar metallicity simulations would reduce this upper limit to $\sim12\,\msun$, likely increasing, as we will see, the number of detectable NSBH mergers.}
%($Z = 0.0002$)
binary population synthesis simulations~\cite{Kruckow_etal:2018} and extending to low masses to reflect the detection of objects in the purported mass gap~\cite{LVC:2020O3acat}. NS masses are drawn from $\prob(\mns) = \uniform(1\msun, 2.42\msun)$, with an upper limit chosen to match that of the DD2 EOS {\bf [which citation?]}, and BH and NS spin magnitudes from the uniform distributions $\prob(\abh) = \uniform(0, 0.99)$ and  $\prob(\ans) = \uniform(0, 0.05)$, assuming they are oriented isotropically. Following Ref.~\cite{Foucart_etal:2018}, we use the component masses, NS radii and BH spins to calculate the baryonic mass ejected by each merger. Note that this formula 1) requires the assumption of a NS EOS (we use DD2) and 2) has been calibrated using simulations without precession. We use the same EOS to calculate tidal deformabilities for the NSs, setting the BH deformabilities to zero. Finally, we generate peculiar velocities for each merger from the Gaussian distribution $\prob(v) = \normal(0\,\kms,500\,\kms)$.

With the ground truth parameters in hand, we generate the data for each merger and apply our selection criteria.  To determine the impact of different physical effects on our results, we simulate two populations using different waveform approximants: the BNS-focused IMRPhenomPv2\_NRTidal~\cite{Dietrich_etal:2019} and NSBH-specific SEOBNRv4\_ROM\_NRTidalv2\_NSBH~\cite{Matas_etal:2020} (hereafter IMRPhenom and SEOBNR). {\bf One does this, the other does that.} As the SEOBNR waveform requires (anti-)aligned spins, we zero the transverse NS and BH spins after sampling them isotropically {\bf [mimicking, in some sense, spins becoming aligned over time?]}. For each waveform, we generate 32-second segments of noisy GW data $\dgw$ per merger\footnote{Using spectra from \url{https://dcc.ligo.org/LIGO-T2000012/public}.}, using a frequency range of 20-2048 Hz and identical random seeds, and consider it detected if the network signal-to-noise ratio is at least $\snrmin = 12$. We assume that the GW detectors operate in concert with an EM followup program capable of detecting all mergers with ejecta mass greater than $\mejmin = 0.01\,M_\odot$; as a result, the selection function is not entirely based on the GW signal. Finally, we generate noisy redshift and peculiar velocity estimates by drawing from $\prob(\zobs|z)=\normal(z,0.001)$ and $\prob(\vobs|v)=\normal(v,200\,\kms)$, respectively.

Of the 25241 simulated mergers, 2477 (2954) are detected in GWs using the IMRPhenom (SEOBNR) waveform, 99 (75) of which have sufficient ejecta to be detected in EM; 62 appear in both samples. {\bf [Why more GW det in SEOBNR?]} Zeroing the transverse spins for use with the SEOBNR waveform has the side-effect of reducing the typical ejecta mass~\cite{Foucart_etal:2018} and hence the final GW+EM-detected sample. The simulated BH masses and spin magnitudes of the SEOBNR population are shown in Figure~\ref{fig:pops} {\bf [relabel to undetected in GW/EM?]}.\footnote{The plot for the IMRPhenom sample is very similar, but with mergers filling the full range of the y-axis as the transverse spins are non-zero.} Here, mergers whose GW SNRs fail (exceed) our detection threshold are displayed as hollow (filled) circles, and mergers with ejecta mass below (above) our EM detection threshold are colored purple (blue). Overlaid are three disruption lines, indicating the BH spins at which the ejecta mass reaches our minimum detection threshold for NS masses equal to the minimum (dot-dashed), mean (solid) and maximum (dashed) of their distribution. The detected population has a clear preference for low BH mass and high BH spin, and thus the number of detections expected will depend strongly on the spin and mass distributions. In particular, we note that populations with near-solar metallicity will likely produce NSBH systems with lower BH masses~\cite{Kruckow_etal:2018} and hence a larger population of GW+EM-detectable mergers.

\begin{figure*}[ht!]
\includegraphics[width=18cm]{{nsbh_pop_H1+_L1+_V1+_K1+_A1_d_32.0_mf_20.0_rf_14.0_dndz_rr_ubhmp_2.5_40.0_unsmp_1.0_2.4_bbhsp_h_0_constraints_binned_by_par}.pdf}
\caption{Distributions of a subset of parameters from our SEOBNR (top) and IMRPhenom (bottom) samples, as drawn from the prior (dotted), selected by GW SNR (dashed) and selected by GW and EM emission (colored histograms). The bins are colored by the fractional \hubble\ uncertainty the mergers within the bin achieve: the yellowest bins are most informative. {\bf To discuss! Distances: SEOBNR GW-selected distances less sharply peaked, typically larger than IMRPhenom. GW+EM bit less clear-cut. Mass ratios: much more important to be low-$q$ for SEOBNR. $z$ spins: preference for positive in both samples (V\& C mention I think?). spin mags: extremal spins more important for IMRPhenom, though this is partially/mostly a prior effect.} \label{fig:pops}}
\end{figure*}

% % % % % % % % % % % % % % % % % % % % % % % % % % % % % %

\section{Methods} \label{sec:methods}

The probabilistic inference of the Hubble constant from catalogues of compact object mergers has been described in detail in the literature~\cite{Schutz:1986,Dalal:2006,Nissanke_etal:2010,Taylor_etal:2012,Nissanke_etal:2013,Abbott_etal:2017a,Chen_etal:2018,Fishbach_etal:2018,Feeney_etal:2018,Mandel_etal:2018,Gray_etal:2019,Mortlock_etal:2019} {\bf [Vitale notes]}. In the following, we adopt a slight variant of the formalism set out in Ref.~\cite{Mortlock_etal:2019}, whose Figure 9 depicts a network diagram for the model we use to describe the data\footnote{The only addition required to the network diagram of~\cite{Mortlock_etal:2019} is the dependence of the selection, $S$, on an intrinsic parameter: the merger's ejecta mass.}. The posterior we evaluate is given by
\begin{widetext}
\begin{align}
\prob \left( \hubble, \decel, \Gamma, \{z, v\} | N, \left\{ \dgw, \hat{z}, \hat{v} \right\}, \snrmin, \mejmin \right) & \propto
\prob \left( \hubble, \decel, \Gamma \right) \exp \left( -\bar{N} \left[ \hubble, \decel, \Gamma, \snrmin, \mejmin \right] \right) \\
& \times \prod_{i = 1}^{N} \frac{\Gamma}{1 + z_i} \frac{{\rm d}V}{{\rm d}z} \left[ \hubble, \decel \right] \prob (\dgw_i | d \left[ z_i, \hubble, \decel \right]) \prob(v_i) \prob(\hat{v}_i | v_i) \prob(\hat{z}_i | z_i), \nonumber
\end{align}
\end{widetext}
%
% one wide line, dropping dependencies
%\begin{widetext}
%\begin{align}
%{\rm P} \left( \hubble, \decel, \Gamma, \{z, v\} | N, \left\{ \hat{\bm{x}}, \hat{z}, \hat{v} \right\}, \snrmin, \mejmin \right) & \propto
%{\rm P} \left( \hubble, \decel, \Gamma \right) e^{-\bar{N}} \prod_{i = 1}^{N} \frac{\Gamma}{1 + z_i} \frac{{\rm d}V}{{\rm d}z} {\rm P} \left( \hat{\bm{x}}_i | d \right) P(v_i) P(\hat{v}_i | v_i) P(\hat{z}_i | z_i),
%\end{align}
%\end{widetext}
%
% dropped dependencies, trying to squeeze into single column
%\begin{align}
%{\rm P} & \left( \hubble, \decel, \Gamma, \{z, v\} | N, \left\{ \hat{\bm{x}}, \hat{z}, \hat{v} \right\}, \snrmin, \mejmin \right) \propto \\
%& {\rm P} \left( \hubble, \decel, \Gamma \right) e^{-\bar{N}} \prod_{i = 1}^{N} \frac{\Gamma}{1 + z_i} \frac{{\rm d}V}{{\rm d}z} {\rm P} \left( \hat{\bm{x}}_i | d \right) P(v_i) P(\hat{v}_i | v_i) P(\hat{z}_i | z_i), \nonumber
%\end{align}
where $N$ and $\nexp$ are the actual and expected number of mergers detected, respectively, curly brackets denote sets of quantities and bold denotes per-merger vectors.

We infer the parameters of this model in two parts, using two sampling methods. We first process each merger individually in order to obtain the GW likelihoods marginalized over all parameters $\boldsymbol{\theta}_i$ other than the luminosity distance $d$ to the merger,
\begin{equation*}
\prob(\dgw_i | d[ z_i, \hubble, \decel ]) = \int {\rm d}\boldsymbol{\theta}_i \prob(\boldsymbol{\theta}_i) \prob(\dgw_i | d[ z_i, \hubble, \decel ], \boldsymbol{\theta}_i),
%\label{eq:marge_like}
\end{equation*}
where
\begin{equation*}
d \simeq \frac{cz}{\hubble} \left[1 + \frac{1}{2} \left(1 - \decel \right) z + \frac{1}{6} \left(-1 + \decel - \jerk + 3 \decel^2 \right) z^2 \right].
%\label{eq:distance}
\end{equation*}
$\boldsymbol{\theta}_i$ comprises the $i^{\rm th}$ merger's component masses, spin magnitudes and orientations (where appropriate), inclination, polarization angle, NS tidal deformability, and time and phase at coalescence. We adopt priors identical to the distributions used in the generative model for all parameters other than the masses. Convergence is greatly improved by sampling chirp masses and mass ratios instead of component masses, and we therefore sample using interim priors that are uniform in these parameters (over the ranges permitted by our component-mass extrema), before importance-sampling the outputs to reinstate our desired component-mass priors. The marginal GW likelihoods are sampled with the \texttt{pypolychord} nested sampler~\cite{Handley_etal:2015a,Handley_etal:2015b}, wrapped by \texttt{bilby}~\cite{Ashton_etal:2019}, using 1000 live points and \texttt{bilby}'s \texttt{marginalize\_phase}, \texttt{\_time} and \texttt{\_distance} settings. Each 15/11-dimensional IMRPhenom (SEOBNR) sampling run takes 6-14 (4-6) days to complete on one Intel Xeon 2.70GHz CPU.

Given the marginal GW likelihoods, we use No-U-Turn Sampling as implemented by the \texttt{pystan} package~\cite{pystan} to infer the cosmological and population parameters. We assume a broad Gaussian prior on \hubble, $\prob(\hubble)=\normal(70,20)$ \kmsmpc, a truncated Gaussian prior on \decel, $\prob(\decel)=\step(\decel+2)\step(1-\decel)\normal(-0.5,0.5)$, and a log-uniform prior on the rate, $\prob(\rate) \propto 1/\rate$. To use \texttt{pystan}, we must be able to sample all parameters from analytic distributions. We therefore perform a Gaussian Mixture Model fit to each merger's marginal distance likelihood using \texttt{pomegranate}~\cite{Schreiber:2017}. We fit each likelihood with an integer grid of 2-10 mixture components, repeating 10 times at each grid point and selecting the best fit using the Akaike Information Criterion~\cite{Akaike:1974}. Finally, we must evaluate the expected number of detected mergers $\nexp$ at each sampled value of the cosmological and population parameters. We do so by re-simulating the catalogues 100 times at each point of a 5x5 grid in $\{H_0,q_0\}$ assuming our fiducial rate, and interpolating the results using a 2D fourth-order interpolant. The dependence on the sampled rate is captured by multiplying the interpolation coefficients by $\Gamma/\Gamma_{\rm fid}$. The resulting 153/193-dimensional \texttt{pystan} inference runs take less than a minute to generate 20,000 well converged samples on a 3.1 GHz Intel Core i7 CPU. The set of true redshifts and peculiar velocities are uninteresting for the purposes of cosmology inference, and we therefore marginalize over these parameters when quoting the results below.

% % % % % % % % % % % % % % % % % % % % % % % % % % % % % %

\section{Results} \label{sec:results}

Processing the simulated SEOBNR and IMRPhenom catalogues through our two-stage inference pipeline produces the cosmology and population parameter posteriors shown in Fig.~\ref{fig:cosmo}. In both cases, the recovered \hubble, \decel\ and rate posteriors are completely consistent with the ground truth values, indicating, as expected, that the selection effects are correctly accounted for. The 68\% credible intervals on the near-Gaussian $\hubble$ marginal posteriors are $68.8 \pm 1.6\,\kmsmpc$ for the SEOBNR sample and $66.7 \pm 1.0\,\kmsmpc$ for IMRPhenom. Recall that the IMRPhenom sample contains 99 objects to the SEOBNR sample's 75, and we should therefore expect that the IMRPhenom sample's $\hubble$ posterior be roughly 13\% narrower than that of the SEOBNR sample. The remaining reduction, therefore, reflects the ability for precessing spins to break the distance-inclination degeneracy {\bf [cites]}. This additional constraining power is equivalent to an approximate doubling of the catalogue size.

The \hubble\ uncertainties we find for both waveforms are comparable to the current Cepheid-SN distance ladder precision. NSBH populations -- should they produce EM counterparts and occur at rates roughly matching our assumptions -- will therefore strongly inform the outcome of the current \hubble\ tension, particularly when combined with accompanying BNS populations, likely of comparable size. The mergers are also informative about the deceleration parameter, $q_0$, shrinking its uncertainty from 0.5 to 0.32 or 0.26, depending on the waveform. This further implies that NSBH catalogues will be able to begin constraining parameters such as the matter density and dark energy EOS (in the context of \lcdm\ and extended models), complementary to BBH results from higher redshifts~\cite{Farr_etal:2019,Chen_etal:2020,Mukherjee_etal:2020}. The merger rates are recovered with roughly 10\% precision.

\begin{figure*}[ht!]
\includegraphics[width=8cm]{{pc_nsbh_pop_H1+_L1+_V1+_K1+_A1_d_32.0_mf_20.0_rf_14.0_dndz_rr_ubhmp_2.5_40.0_unsmp_1.0_2.4_bbhsp_seobnr_aligned_gmm_fits_rate_cosmo_post_triangle_plot}.pdf}\includegraphics[width=8cm]{{pc_nsbh_pop_H1+_L1+_V1+_K1+_A1_d_32.0_mf_20.0_rf_14.0_dndz_rr_ubhmp_2.5_40.0_unsmp_1.0_2.4_bbhsp_gmm_fits_rate_cosmo_post_triangle_plot}.pdf}
\caption{Cosmological and population parameter posteriors inferred for the simulated SEOBNR (left) and IMRPhenom (right) NSBH samples.\label{fig:cosmo}}
\end{figure*}

Let's now look at the results for individual mergers in closer detail, starting with the RHS of Fig.~\ref{fig:pops}. This shows the typical constraints on individual mergers' BH masses and spin magnitudes when using the SEOBNR waveform. A selection of zoomed-in versions of these constraints for both the SEOBNR (red, filled) and IMRPhenom (grey, filled) waveforms can be found in the top row of Fig.~\ref{fig:waveforms}. The selection is the highest SNR event common to both samples (left), the IMRPhenom event whose spins are closest to aligned (right), and three other mergers whose posteriors illustrate the types of degeneracies present. These plots bode well for recovering information about the neutron-star equation of state from the position of the tidal disruption line, though this is the focus of future work. Information more pertinent to the question at hand can be found in the bottom row of Fig.~\ref{fig:waveforms}, in which we show the constraints on the distance and inclination of a selection of mergers, whose various signal-to-noise ratios are indicated in the top row. What is clear, here, is the importance of precession in breaking the distance-inclination degeneracy, particularly in high-SNR events. We illustrate this further by re-running the IMRPhenom cases assuming aligned spins, having set their x and y spin components to zero. These results are overlaid as dashed dark red contours. The distance-inclination degeneracies blow up for the highest-SNR events, which will have a clear impact on the population's ability to constrain $H_0$.

\begin{figure}[ht!]
\includegraphics[width=8cm]{{nsbh_pop_H1+_L1+_V1+_K1+_A1_d_32.0_mf_20.0_rf_14.0_dndz_rr_ubhmp_2.5_40.0_unsmp_1.0_2.4_bbhsp_frac_errs}.pdf}
\caption{Top: joint distance and inclination posteriors for the highest signal-to-noise merger common to both detected populations, simulated and sampled using the IMRPhenom waveform with precessing (grey filled) and aligned (dark red dashed) spins, and using the SEOBNR waveform (red) with aligned spins. Bottom: distributions of fractional uncertainties on luminosity distance (dotted) and \hubble\ (solid) from individual mergers from our IMRPhenom (grey) and SEOBNR (red) NSBH samples. \label{fig:waveforms}}
\end{figure}

% % % % % % % % % % % % % % % % % % % % % % % % % % % % % %

\section{Conclusions} \label{sec:conclusions}
Unbiased. Comparable to Cepheids. Particular physics / waveform matters in detail: precession buys you factor of two in sample size. Will inform \hubble\ debate.

% % % % % % % % % % % % % % % % % % % % % % % % % % % % % %

\begin{acknowledgments}
S.M. Feeney is supported by the Royal Society.
Author contributions. https://www.elsevier.com/authors/journal-authors/policies-and-ethics/credit-author-statement
\end{acknowledgments}

% % % % % % % % % % % % % % % % % % % % % % % % % % % % % %

\bibliography{references}% Produces the bibliography via BibTeX.

\end{document}
%
% ****** End of file apssamp.tex ******