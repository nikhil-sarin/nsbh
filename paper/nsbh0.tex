
%% Beginning of file 'sample63.tex'
%%
%% Modified 2019 June
%%
%% This is a sample manuscript marked up using the
%% AASTeX v6.3 LaTeX 2e macros.
%%
%% AASTeX is now based on Alexey Vikhlinin's emulateapj.cls 
%% (Copyright 2000-2015).  See the classfile for details.

%% AASTeX requires revtex4-1.cls (http://publish.aps.org/revtex4/) and
%% other external packages (latexsym, graphicx, amssymb, longtable, and epsf).
%% All of these external packages should already be present in the modern TeX 
%% distributions.  If not they can also be obtained at www.ctan.org.

%% The first piece of markup in an AASTeX v6.x document is the \documentclass
%% command. LaTeX will ignore any data that comes before this command. The 
%% documentclass can take an optional argument to modify the output style.
%% The command below calls the preprint style which will produce a tightly 
%% typeset, one-column, single-spaced document.  It is the default and thus
%% does not need to be explicitly stated.
%%
%%
%% using aastex version 6.3
\documentclass[twocolumn]{aastex63}
\usepackage{amsmath,bm}

%% The default is a single spaced, 10 point font, single spaced article.
%% There are 5 other style options available via an optional argument. They
%% can be invoked like this:
%%
%% \documentclass[arguments]{aastex63}
%% 
%% where the layout options are:
%%
%%  twocolumn   : two text columns, 10 point font, single spaced article.
%%                This is the most compact and represent the final published
%%                derived PDF copy of the accepted manuscript from the publisher
%%  manuscript  : one text column, 12 point font, double spaced article.
%%  preprint    : one text column, 12 point font, single spaced article.  
%%  preprint2   : two text columns, 12 point font, single spaced article.
%%  modern      : a stylish, single text column, 12 point font, article with
%% 		  wider left and right margins. This uses the Daniel
%% 		  Foreman-Mackey and David Hogg design.
%%  RNAAS       : Preferred style for Research Notes which are by design 
%%                lacking an abstract and brief. DO NOT use \begin{abstract}
%%                and \end{abstract} with this style.
%%
%% Note that you can submit to the AAS Journals in any of these 6 styles.
%%
%% There are other optional arguments one can invoke to allow other stylistic
%% actions. The available options are:
%%
%%   astrosymb    : Loads Astrosymb font and define \astrocommands. 
%%   tighten      : Makes baselineskip slightly smaller, only works with 
%%                  the twocolumn substyle.
%%   times        : uses times font instead of the default
%%   linenumbers  : turn on lineno package.
%%   trackchanges : required to see the revision mark up and print its output
%%   longauthor   : Do not use the more compressed footnote style (default) for 
%%                  the author/collaboration/affiliations. Instead print all
%%                  affiliation information after each name. Creates a much 
%%                  longer author list but may be desirable for short 
%%                  author papers.
%% twocolappendix : make 2 column appendix.
%%   anonymous    : Do not show the authors, affiliations and acknowledgments 
%%                  for dual anonymous review.
%%
%% these can be used in any combination, e.g.
%%
%% \documentclass[twocolumn,linenumbers,trackchanges]{aastex63}
%%
%% AASTeX v6.* now includes \hyperref support. While we have built in specific
%% defaults into the classfile you can manually override them with the
%% \hypersetup command. For example,
%%
%% \hypersetup{linkcolor=red,citecolor=green,filecolor=cyan,urlcolor=magenta}
%%
%% will change the color of the internal links to red, the links to the
%% bibliography to green, the file links to cyan, and the external links to
%% magenta. Additional information on \hyperref options can be found here:
%% https://www.tug.org/applications/hyperref/manual.html#x1-40003
%%
%% Note that in v6.3 "bookmarks" has been changed to "true" in hyperref
%% to improve the accessibility of the compiled pdf file.
%%
%% If you want to create your own macros, you can do so
%% using \newcommand. Your macros should appear before
%% the \begin{document} command.
%%
\newcommand{\vdag}{(v)^\dagger}
\newcommand\aastex{AAS\TeX}
\newcommand\latex{La\TeX}

%% Reintroduced the \received and \accepted commands from AASTeX v5.2
%\received{June 1, 2019}
%\revised{January 10, 2019}
%\accepted{\today}
\received{}
\revised{}
\accepted{}
%% Command to document which AAS Journal the manuscript was submitted to.
%% Adds "Submitted to " the argument.
\submitjournal{ApJL}

%%%%%%%%%%%%%%%%%%%%%%%%%%%%%%%%%%%%%%%%%%%%%%%%%%%%%%%%%%%%%%%%%%%%%%%%%%%%%%%%
%%
%% The following section outlines numerous optional output that
%% can be displayed in the front matter or as running meta-data.
%%
%% If you wish, you may supply running head information, although
%% this information may be modified by the editorial offices.
\shorttitle{NSB$H_0$}
\shortauthors{Feeney et al.}
%%
%% You can add a light gray and diagonal water-mark to the first page 
%% with this command:
%% \watermark{text}
%% where "text", e.g. DRAFT, is the text to appear.  If the text is 
%% long you can control the water-mark size with:
%% \setwatermarkfontsize{dimension}
%% where dimension is any recognized LaTeX dimension, e.g. pt, in, etc.
%%
%%%%%%%%%%%%%%%%%%%%%%%%%%%%%%%%%%%%%%%%%%%%%%%%%%%%%%%%%%%%%%%%%%%%%%%%%%%%%%%%

%% This is the end of the preamble.  Indicate the beginning of the
%% manuscript itself with \begin{document}.

\begin{document}

\title{Measuring the Hubble Constant with Neutron-Star-Black-Hole Mergers in the A+ Era}

\correspondingauthor{Stephen M. Feeney}
\email{stephen.feeney@ucl.ac.uk}

\author[0000-0003-2268-2519]{Stephen M. Feeney}
\affiliation{Department of Physics \& Astronomy, University College London, Gower Street, London WC1E 6BT, UK}
\author[0000-0002-0041-3783]{Daniel J. Mortlock}
\affiliation{Astrophysics Group, Imperial College London, Blackett Laboratory, Prince Consort Road, London SW7 2AZ, UK}
\affiliation{Department of Mathematics, Imperial College London, London SW7 2AZ, UK}
\affiliation{Department of Astronomy, Stockholm University, AlbaNova, SE-10691 Stockholm, Sweden}
\author[0000-0001-6573-7773]{Samaya M. Nissanke}
\affiliation{GRAPPA, Anton Pannekoek Institute for Astronomy and Institute of High-Energy Physics, University of Amsterdam, Science Park 904, 1098 XH Amsterdam, The Netherlands}
\affiliation{Nikhef, Science Park 105, 1098 XG Amsterdam, The Netherlands}
\author[0000-0002-2519-584X]{Hiranya V. Peiris}
\affiliation{Department of Physics \& Astronomy, University College London, Gower Street, London WC1E 6BT, UK}
\affiliation{Oskar Klein Centre for Cosmoparticle Physics, Department of Physics,
Stockholm University, AlbaNova, Stockholm SE-106 91, Sweden}

%% Mark off the abstract in the ``abstract'' environment. 
\begin{abstract}

The Astrophysical Journal Letters (ApJL) has a 250 word limit for the abstract.

\end{abstract}

%% Keywords should appear after the \end{abstract} command. 
%% See the online documentation for the full list of available subject
%% keywords and the rules for their use.
\keywords{editorials, notices --- 
miscellaneous --- catalogs --- surveys}

%% From the front matter, we move on to the body of the paper.
%% Sections are demarcated by \section and \subsection, respectively.
%% Observe the use of the LaTeX \label
%% command after the \subsection to give a symbolic KEY to the
%% subsection for cross-referencing in a \ref command.
%% You can use LaTeX's \ref and \label commands to keep track of
%% cross-references to sections, equations, tables, and figures.
%% That way, if you change the order of any elements, LaTeX will
%% automatically renumber them.
%%
%% We recommend that authors also use the natbib \citep
%% and \citet commands to identify citations.  The citations are
%% tied to the reference list via symbolic KEYs. The KEY corresponds
%% to the KEY in the \bibitem in the reference list below. 

\section{Introduction} \label{sec:intro}

Other restrictions:
\begin{enumerate}
	\item Main Text -- no more than 3500 words (not including appendices or other supplementary material)
	\item Figures and Tables -- no more than 5 combined figures (each limited to 9 panels) and tables, e.g. 3 figures and 2 tables.
	\item References -- no more than 50 references
\end{enumerate}


\section{Methods} \label{sec:methods}

The final posterior we evaluate is given by
\begin{widetext}
\begin{align}
{\rm P}(H_0, q_0, \Gamma, \{z, v\} | N, \{\hat{\bm{h}}_+, \hat{\bm{h}}_\times, \hat{z}, \hat{v} \}, \rho_*, I) & \propto
{\rm P}(H_0, q_0, \Gamma | I) \exp \left( -\bar{N}[H_0, q_0, \Gamma, \rho_*] \right) \\
& \times \prod_{i = 1}^{N} \frac{\Gamma}{1 + z_i} \frac{{\rm d}V}{{\rm d}z}(H_0, q_0) {\rm P}(\hat{\bm{h}}_{+,i}, \hat{\bm{h}}_{\times,i} | d_L[z_i, H_0, q_0]) P(v_i|I) P(\hat{v}_i | v_i) P(\hat{z}_i | z_i), \nonumber
\end{align}
\end{widetext}
where I've dropped a bunch of fixed parameters, such as the noise curves, errors on observed peculiar velocities and redshifts, observation time (perhaps we can bundle all of these into a set $O$?). Brackets denote sets of quantities, bold denotes per-source vectors. For the purposes of learning about cosmology, the true redshifts and peculiar velocities are uninteresting, and so we marginalize over these parameters. The first PDF in the product is the gravitational-wave likelihood marginalized over all parameters ($\boldsymbol{\theta}$) other than the distance to the merger,
\begin{equation}
{\rm P}(\hat{\bm{h}}_{+,i}, \hat{\bm{h}}_{\times,i} | d_L) = \int {\rm d}\boldsymbol{\theta} \, {\rm P}(\boldsymbol{\theta},I) {\rm P}(\hat{\bm{h}}_{+,i}, \hat{\bm{h}}_{\times,i} | d_L, \boldsymbol{\theta}),
\end{equation}
where $\boldsymbol{\theta}$ consists of the merger's chirp mass, mass ratio, spin magnitudes and orientations (where appropriate), inclination, polarization angle, tidal deformability, and time and phase at coalescence.


\section{Results} \label{sec:results}

Some plots, which make the acknowledgments go nuts.

\begin{figure*}[ht!]
\plotone{{pc_nsbh_pop_H1+_L1+_V1+_K1+_A1_d_32.0_mf_20.0_rf_14.0_dndz_rr_ubhmp_2.5_40.0_unsmp_1.0_2.4_bbhsp_seobnr_aligned_imp_sample_weighted_samples_disruption_line_in_out_comp}.pdf}
\caption{Blah.\label{fig:pops}}
\end{figure*}

\begin{figure*}[ht!]
\plotone{{nsbh_pop_H1+_L1+_V1+_K1+_A1_d_32.0_mf_20.0_rf_14.0_dndz_rr_ubhmp_2.5_40.0_unsmp_1.0_2.4_bbhsp_seobnr_aligned_imp_sample_weighted_samples_spin_z_mass_dis_inc_wf_comp}.pdf}
\caption{Posterior distributions for mergers simulated using IMRPhenom (blue) and SEOBNR (pink) waveforms. Each column depicts the distance, inclination, and black-hole mass and $z$-spin magnitude posteriors for a merger with true parameters indicated by black plus symbols (crosses where the true $z$-spin is negative). Shown are: (left) the merger with the highest signal-to-noise, (right) the two IMRPhenom mergers whose black-hole spins are closest to being aligned, and (center) two further mergers indicating the range of posteriors observed. \label{fig:waveforms}}
\end{figure*}

\begin{figure*}[ht!]
\plottwo{{pc_nsbh_pop_H1+_L1+_V1+_K1+_A1_d_32.0_mf_20.0_rf_14.0_dndz_rr_ubhmp_2.5_40.0_unsmp_1.0_2.4_bbhsp_seobnr_aligned_gmm_fits_rate_cosmo_post_triangle_plot}.pdf}{{pc_nsbh_pop_H1+_L1+_V1+_K1+_A1_d_32.0_mf_20.0_rf_14.0_dndz_rr_ubhmp_2.5_40.0_unsmp_1.0_2.4_bbhsp_gmm_fits_rate_cosmo_post_triangle_plot}.pdf}
\caption{Blah.\label{fig:cosmo}}
\end{figure*}


\section{Conclusions} \label{sec:conclusions}


\acknowledgments

We thank some people.

%% To help institutions obtain information on the effectiveness of their 
%% telescopes the AAS Journals has created a group of keywords for telescope 
%% facilities.
%
%% Following the acknowledgments section, use the following syntax and the
%% \facility{} or \facilities{} macros to list the keywords of facilities used 
%% in the research for the paper.  Each keyword is check against the master 
%% list during copy editing.  Individual instruments can be provided in 
%% parentheses, after the keyword, but they are not verified.

\vspace{5mm}

%% Similar to \facility{}, there is the optional \software command to allow 
%% authors a place to specify which programs were used during the creation of 
%% the manuscript. Authors should list each code and include either a
%% citation or url to the code inside ()s when available.

\software{numpy \citep{}, matplotlib \citep{}, lalsim \citep{}, bilby \citep{}, pypolychord \citep{}, pomegranate \citep{}, pystan \citep{}, getdist \citep{}}

%% Appendix material should be preceded with a single \appendix command.
%% There should be a \section command for each appendix. Mark appendix
%% subsections with the same markup you use in the main body of the paper.

%% Each Appendix (indicated with \section) will be lettered A, B, C, etc.
%% The equation counter will reset when it encounters the \appendix
%% command and will number appendix equations (A1), (A2), etc. The
%% Figure and Table counter will not reset.

\appendix

\section{Appendix information}

An appendix.

\bibliography{sample63}{}
\bibliographystyle{aasjournal}

%% This command is needed to show the entire author+affiliation list when
%% the collaboration and author truncation commands are used.  It has to
%% go at the end of the manuscript.
%\allauthors

%% Include this line if you are using the \added, \replaced, \deleted
%% commands to see a summary list of all changes at the end of the article.
%\listofchanges

\end{document}

% End of file `sample63.tex'.
