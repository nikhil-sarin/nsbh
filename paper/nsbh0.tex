% ****** Start of file apssamp.tex ******
%
%   This file is part of the APS files in the REVTeX 4.2 distribution.
%   Version 4.2a of REVTeX, December 2014
%
%   Copyright (c) 2014 The American Physical Society.
%
%   See the REVTeX 4 README file for restrictions and more information.
%
% TeX'ing this file requires that you have AMS-LaTeX 2.0 installed
% as well as the rest of the prerequisites for REVTeX 4.2
%
% See the REVTeX 4 README file
% It also requires running BibTeX. The commands are as follows:
%
%  1)  latex apssamp.tex
%  2)  bibtex apssamp
%  3)  latex apssamp.tex
%  4)  latex apssamp.tex
%
\documentclass[%
 reprint,
 superscriptaddress,
%groupedaddress,
%unsortedaddress,
%runinaddress,
%frontmatterverbose, 
%preprint,
%preprintnumbers,
 nofootinbib,
%nobibnotes,
%bibnotes,
 amsmath,amssymb,
 aps,
%pra,
%prb,
%rmp,
%prstab,
%prstper,
%floatfix,
]{revtex4-2}

\usepackage{graphicx}% Include figure files
\usepackage{dcolumn}% Align table columns on decimal point
\usepackage{bm}% bold math
\usepackage{hyperref}% add hypertext capabilities
\usepackage{xcolor}

%\usepackage[showframe,%Uncomment any one of the following lines to test 
%%scale=0.7, marginratio={1:1, 2:3}, ignoreall,% default settings
%%text={7in,10in},centering,
%%margin=1.5in,
%%total={6.5in,8.75in}, top=1.2in, left=0.9in, includefoot,
%%height=10in,a5paper,hmargin={3cm,0.8in},
%]{geometry}

% % % % % % % % % % % % % % % % % % % % % % % % % % % % % %

% journals
\newcommand{\mnras}{MNRAS}
\newcommand{\aap}{A\&A}
\newcommand{\aaps}{A\&AS}
\newcommand{\apjs}{ApJS}
\newcommand{\apjl}{ApJL}
\newcommand{\araa}{ARA\&A}
\newcommand{\jhep}{JHEP}
\newcommand{\aj}{AJ}
\newcommand{\jcap}{JCAP}
\newcommand{\pasp}{PASP}
\newcommand{\pnas}{PNAS}
\newcommand{\epjc}{EPJC}

% maths
%\newcommand{\dataset}{\vect{d}}
\newcommand{\msun}{M_\odot}
\newcommand{\hubble}{\ensuremath{H_0}}
\newcommand{\hubbleest}{\ensuremath{\hat{H}_0}}
\newcommand{\decel}{\ensuremath{q_0}}
\newcommand{\jerk}{\ensuremath{j_0}}
\newcommand{\dl}{\ensuremath{D}}
\newcommand{\vobs}{\ensuremath{\hat{v}}}
\newcommand{\vobss}{\ensuremath{\hat{\vect{v}}}}
\newcommand{\zobs}{\ensuremath{\hat{z}}}
\newcommand{\zmax}{\ensuremath{z_{\rm max}}}
\newcommand{\prob}{\ensuremath{{\rm P}}}
\newcommand{\normal}{{\rm{N}}}
\newcommand{\sels}{\vect{S}}
\newcommand{\inc}{\iota}
\newcommand{\nexp}{\bar{N}}
\newcommand{\abh}{a_{\rm BH}}
\newcommand{\ans}{a_{\rm NS}}
\newcommand{\mbh}{m_{\rm BH}}
\newcommand{\mns}{m_{\rm NS}}
\newcommand{\mchirp}{{\cal{M}}}
\newcommand{\strain}{h}
\newcommand{\strainobs}{\hat{h}}
\newcommand{\sigmav}{\sigma_{||}}
\newcommand{\gwpoppar}{\beta}
\newcommand{\uniform}{{\rm U}}
\newcommand{\tobs}{t_{\rm obs}}
\newcommand{\fobs}{f_{\rm obs}}
\newcommand{\rate}{\Gamma}
\newcommand{\step}{\Theta}
\newcommand{\snr}{\rho}
\newcommand{\snrmin}{\rho_*}
\newcommand{\mejmin}{m_{\rm ej}^*}
\newcommand{\dgw}{\hat{\bm{x}}}
\newcommand{\kmsmpc}{\ensuremath{{\rm km\,s^{-1}\,Mpc^{-1}}}}
\newcommand{\kms}{\ensuremath{{\rm km\,s^{-1}}}}
\newcommand{\mpc}{\ensuremath{{\rm Mpc}}}
\newcommand{\gpc}{\ensuremath{{\rm Gpc}}}
\newcommand{\yr}{\ensuremath{{\rm yr}}}
\newcommand{\yrgpc}{\ensuremath{{\rm yr^{-1}\,Gpc^{-3}}}}
\newcommand{\planck}{{\it Planck}}
\newcommand{\plancks}{{\it Planck{\rm 's}}}
\newcommand{\lcdm}{$\Lambda$CDM}

% waveforms
\newcommand{\seobnr}{\texttt{SEOBNR}}
\newcommand{\seobnrfull}{\texttt{SEOBNRv4\_ROM\_NRTidalv2\_NSBH}}
\newcommand{\imrp}{\texttt{IMRPhenom}}
\newcommand{\imrpfull}{\texttt{IMRPhenomPv2\_NRTidal}}

% editing
\newcommand{\smf}[1]{\textcolor{red}{\bf [#1]}}
\newcommand{\samaya}[1]{\textcolor{cyan}{\bf [#1]}}
\newcommand{\samayacomments}[1]{\textcolor{blue}{\bf [#1]}}
% % % % % % % % % % % % % % % % % % % % % % % % % % % % % %

\begin{document}

%\preprint{APS/123-QED}

\title{Prospects for Measuring the Hubble Constant with Neutron-Star-Black-Hole Mergers in the A+ Era}

\author{Stephen M. Feeney}
\affiliation{Department of Physics \& Astronomy, University College London, Gower Street, London WC1E 6BT, UK}
\author{Hiranya V. Peiris}
\affiliation{Department of Physics \& Astronomy, University College London, Gower Street, London WC1E 6BT, UK}
\affiliation{Oskar Klein Centre for Cosmoparticle Physics, Department of Physics,
Stockholm University, AlbaNova, Stockholm SE-106 91, Sweden}
\author{Samaya M. Nissanke}
\affiliation{GRAPPA, Anton Pannekoek Institute for Astronomy and Institute of High-Energy Physics, University of Amsterdam, Science Park 904, 1098 XH Amsterdam, The Netherlands}
\affiliation{Nikhef, Science Park 105, 1098 XG Amsterdam, The Netherlands}
\author{Daniel J. Mortlock}
\affiliation{Astrophysics Group, Imperial College London, Blackett Laboratory, Prince Consort Road, London SW7 2AZ, UK}
\affiliation{Department of Mathematics, Imperial College London, London SW7 2AZ, UK}
\affiliation{Department of Astronomy, Stockholm University, AlbaNova, SE-10691 Stockholm, Sweden}

\date{\today}% It is always \today, today,
             %  but any date may be explicitly specified

% % % % % % % % % % % % % % % % % % % % % % % % % % % % % %

\begin{abstract}
%The current controversy over the expansion rate of the Universe, \hubble, requires a precise and accurate local verification for resolution. Combined gravitational-wave (GW) and electromagnetic (EM) observations of nearby compact-object mergers are a particularly promising prospect, yielding \hubble\ estimates that depend on general relativity only. The utility of binary neutron star (BNS) mergers has already been proven by GW170817/GRB170817A, but the potential for as-yet undiscovered neutron-star-black-hole (NSBH) mergers to contribute is just beginning to be explored. Using idealized, fixed-signal-to-noise simulations at indicative parameter values, Vitale \& Chen~\cite{Vitale_Chen:2018} demonstrated that catalogues of GW-selected NSBH observations can constrain \hubble\ as well as (or better than) BNSs, depending on the relative merger rates and black hole spins. Here, we perform the first end-to-end simulations of NSBH samples, using fully specified parent populations, a complete noise treatment, and combined GW and EM selection, to determine the \hubble\ constraints realistic samples will achieve by the ``A+'' era of the mid-2020s. Using two waveforms to account for the uncertainty in the modeling of NSBH signals, we show that NSBH samples will yield unbiased \hubble\ estimates with up to 1.6 to 1.0 \kmsmpc\ precision, dependent on whether these systems have non-precessing or precessing spins. This level of precision is ideal for resolving the \hubble\ tension. The degree to which the spins precess will be critical: 
%%, accounting for an effective doubling in the sample size compared to non-precessing spins.
%thanks to their significantly reduced distance-inclination degeneracies, we find precessing-spin samples to be equivalent to non-precessing samples of twice the size.
Gravitational wave (GW) and electromagnetic (EM) observations of neutron-star-black-hole (NSBH) mergers have the potential to provide the precise, accurate and direct verification of the Hubble constant (\hubble) cosmology urgently needs. We perform the first end-to-end analysis of simulated NSBHs, incorporating both GW and EM selection, to determine that NSBHs should achieve unbiased 1-1.6 km/s/Mpc precision estimates by the mid-2020s: ideal for resolving the \hubble\ tension. The degree to which the systems' spins precess is critical, with significant precession equivalent to doubling a non-precessing sample's size. {\bf [614/600 chars]}
\end{abstract}

% % % % % % % % % % % % % % % % % % % % % % % % % % % % % %

%\keywords{Suggested keywords}%Use showkeys class option if keyword
                              %display desired
\maketitle

% % % % % % % % % % % % % % % % % % % % % % % % % % % % % %

\section{Introduction} \label{sec:intro}

%PRL maximum length: 3750 words; abstract should be under 600 characters.

The current expansion rate of the Universe -- the Hubble constant, or $H_0$ -- is at the heart of a significant cosmological controversy. Measured directly in the local Universe by the SH$_0$ES team's Cepheid-supernova distance ladder~\cite{Riess_etal:2019}, it is found to be $74.03 \pm 1.42 \,\kmsmpc$: discrepant at the 4.4-$\sigma$ level with the $67.36 \pm 0.54\,\kmsmpc$ inferred from the \planck\ satellite's observations of the cosmic microwave background (CMB) anisotropies, assuming the standard cosmological model~\cite{Planck_VI:2018}.

There are two potential explanations for this discrepancy, the most exciting of which derives from the model-dependence of the CMB constraint: could the discrepancy be due to physics beyond the standard model? The field has generated no shortage of extended models aiming to bring the two measurements into agreement (see, e.g., Ref.~\cite{Knox_Millea:2020} for a review of the relevant physics, and Ref.~\cite{Vagnozzi:2020} for a near-exhaustive list of references), but consensus on a compelling theoretical explanation is yet to be reached. The second, more prosaic explanation is that there are undiagnosed systematic errors in the CMB or distance ladder data. Again, though, despite multiple investigations of both the CMB~\cite{Spergel_etal:2015,Addison_etal:2016,Obied_etal:2017,Calabrese_etal:2017,Efstathiou_Gratton:2019,Motloch_Hu:2020,ACT:2020} \smf{Add BeyondPlanck XII if it appears} and distance ladder~\cite{Efstathiou:2014,Rigault_etal:2015,Jones_etal:2015,Cardona_etal:2016,Zhang_etal:2017,Follin_Knox:2017,Feeney_etal:2017,Wu_Huterer:2017,Dhawan_etal:2017,Bengaly_etal:2018,Rigault_etal:2018,Jones_etal:2018,Riess_etal:2020,Efstathiou:2020} datasets, no study has found incontrovertible evidence warranting a change of conclusions.

In the absence of conclusive evidence of systematic errors or consensus on an extended model, independent verifications of the two central measurements offer a promising route to resolving the tension. Compelling verification of the CMB anisotropy constraints comes from recent inverse distance ladder datasets~\cite{Addison_etal:2017,DES_H_0:2017,Philcox_etal:2020}, which constrain \hubble\ to $\sim$$68 \pm 1\,\kmsmpc$ by combining big bang nucleosynthesis, CMB {\it intensity}, baryon acoustic oscillation and galaxy clustering measurements. Local verification has proven more challenging, with some alternative analyses supporting the SH$_0$ES team's findings~\cite{Yuan_etal:2019,Huang_etal:2020,H0LICOW_XIII:2020,TDCOSMO_I:2020,Pesce_etal:2020} and others providing contradictory conclusions of varying significance~\cite{Freedman_etal:2019,Freedman_etal:2020,TDCOSMO_IV:2020,Boruah_etal:2020}, in some cases using the same data.

It is clear that a direct, completely independent local measurement with percent-level precision is needed to resolve the \hubble\ tension. Combined gravitational-wave (GW) and electromagnetic (EM) observations of nearby compact-object mergers are ideal candidates to provide that measurement, yielding \hubble\ estimates that depend on general relativity alone~\cite{Schutz:1986,Holz_Hughes:2005,Dalal:2006,Nissanke_etal:2010,Taylor_etal:2012,Messenger_Read:2012,Nissanke_etal:2013,Oguri:2016,delPozzo:2017,Abbott_etal:2017a,Seto:2018,Chen_etal:2018,Fishbach_etal:2018,Feeney_etal:2018,Mortlock_etal:2019,Soares-Santos_etal:2019,Gray_etal:2019,Palmese_etal:2020,Vasylyev_Filippenko:2020,Chen_etal:2020,Gayathri_etal:2020,Mukherjee_etal:2020}. Thanks to their accompanying EM emission, the utility of binary neutron star (BNS) mergers is well established~\cite{Dalal:2006,Nissanke_etal:2010,Taylor_etal:2012,Messenger_Read:2012,Nissanke_etal:2013,Oguri:2016,delPozzo:2017,Abbott_etal:2017a,Seto:2018,Chen_etal:2018,Fishbach_etal:2018,Feeney_etal:2018,Mortlock_etal:2019,Gray_etal:2019}, but the potential for as-yet undiscovered neutron-star-black-hole (NSBH) mergers with EM counterparts to contribute has received much less attention~\cite{Nissanke_etal:2010,Nissanke_etal:2013,Vitale_Chen:2018}. Using idealized, fixed-signal-to-noise simulations at indicative parameter values,~\citet{Vitale_Chen:2018} recently showed that catalogues of GW-selected NSBH observations may constrain \hubble\ as well as (or better than) BNSs, depending on the relative merger rates and black hole spins. \samayacomments{not sure I would add better than due to merger rates as well as only a subset from EM counterparts}. \samaya{In particular, \citet{Vitale_Chen:2018} showed explicitly that luminosity distance measures could improve because misaligned BH spins induce spin precession and may contribute to degeneracy breaking between the luminosity distance and inclination angle for some NSBH systems.} Here, we perform the first end-to-end simulations of NSBH samples, using fully specified parent populations, combined GW and EM selection, and a complete noise treatment, to determine the \hubble\ constraints realistic NSBH samples will achieve by the ``A+'' era of the mid-2020s. \samaya{In particular, for GW and EM selection, we use most recent GW waveforms for NS-BH (CITE) and EM outflow models (CITE Foucart+2018) respectively, both of which have been calibrated to a suite of numerical relativity simulations. In addition, we assume an expanded GW network with the addition of LIGO India and KAGRA.}  \smf{Samaya, can you bulk this out please? Presumably we should also mention our use of multiple waveforms?}

%The current expansion rate of the Universe -- the Hubble constant, or $H_0$ -- is at the center of a serious cosmological controversy. Measured directly in the local Universe by the SH0ES team's Cepheid-supernova distance ladder, it is found to be blah. Estimated in a cosmological-model-dependent fashion from the Planck satellite's observations of the cosmic microwave background's temperature and polarization anisotropies, it is found to be wah. This is a big ol' discrepancy.
%
%There are two potential explanations here. The most exciting of the two derives from the model-dependence of the CMB constraint: could this discrepancy be due to novel physics missing from the $\Lambda$CDM model used in the Planck analysis? This has, understandably, let to a great amount of activity among theorists [cites], though consensus on a compelling theoretical explanation is yet to be reached. Arguably the strongest (least weak?) evidence points at modifications to the expansion history in the decade of scale factor preceding recombination [EDE, Knox\& Millea] (though see [hill etc.] for recent contradictory evidence).
%
%The second, more prosaic explanation, is that there are undiagnosed analysis issues in the CMB or distance ladder data. There are two approaches here, both of which have been heavily undertaken by the field: search for undiagnosed systematic errors in the constituent datasets, and harness complementary independent datasets to verify the conclusions of the low- and high-redshift universes. On the systematics side, there have been loads of investigations into the Cepheid-SN distance ladder [Cardona, Follin, Feeney, Rigault and responses] and the CMB [Spergel, more?], none of which have found incontrovertible evidence warranting a change of conclusions. On the new data side of things, the high-z Universe is pretty much sorted: the inverse distance ladder agrees with the CMB very well, which is particularly convincing given entirely CMB-independent formulations. Low-z is a little less clear? Various distance ladders (MIRAs, tRGB controversy) and strong lensing (though again, controversy) generally back up the Cepheids. Clear that a completely independent measurement with same precision would be enormously helpful.
%
%Here's where GWs come in. Combining GW detections, which yield distance estimates, with EM counterparts, it's possible to estimate $H_0$ assuming general relativity alone [Schutz, Holz, Samaya]. With the advent of detections from LIGO and Virgo, starting to do this, both directly in the one case where the counterpart is known (LVC) and statistically in the case where it is not (many). Prime candidate here is BNS, as counterparts expected and things much more precise [us x 2, chen]. BBH can also be useful (see recent papers on blah, but note association is not firm, and model-dependent). What hasn't been researched so much is NSBH. Discuss [Chen Vitale], and motivate what we're doing here. Full ranges of parameters. Different waveforms. EM selection. This is the focus of this paper. [Vasylyev, Alex Filippenko dark measurement?]

% % % % % % % % % % % % % % % % % % % % % % % % % % % % % %

\section{Simulations} \label{sec:sims}

In this work we simulate the results of a circa-2025 GW detector network, consisting of LIGO A+, Virgo AdV+, KAGRA and LIGO India~\cite{Abbott_etal:2013} \smf{Samaya, what's the best way to cite Sukanta Bose private comm / LIGO-T2000158} observing for $\tobs = 5 \yr$ with a 50\% duty cycle, $\fobs$. We assume a constant rest-frame  NSBH merger rate $\Gamma = 610$ \yrgpc\ (matching the 90\% upper limit of Ref.~\cite{Ligo:2018}), and ground truth cosmological parameters matching Ref.~\cite{Planck_VI:2018}, with $\hubble=67.36\,\kmsmpc$ and $\decel=-0.527$.

%H1+ LVC_AplusDesign_PSD.txt https://dcc.ligo.org/LIGO-T2000012/public
%L1+ LVC_AplusDesign_PSD.txt https://dcc.ligo.org/LIGO-T2000012/public
%V1+ LVC_avirgo_O5high_NEW_PSD.txt https://dcc.ligo.org/LIGO-T2000012/public
%K1+ LVC_kagra_128Mpc_PSD.txt https://arxiv.org/pdf/1102.5421.pdf
%A1 LVC_AplusDesign_PSD.txt

Each simulation starts by drawing the total number of mergers from a Poisson distribution with mean $\lambda = \fobs \, \tobs \, V \, \Gamma $, where $V = 4\pi \int_0^{\zmax} \frac{{\rm d}V}{{\rm d}z} \frac{{\rm d}z}{1+z}$ is the redshifted volume and we use a cosmographic co-moving volume element, expanded to third order (the jerk, $j_0$, which we take to be one).\footnote{\begin{equation}
\frac{{\rm d}V}{{\rm d}z} \simeq 4\pi \frac{c^3 z^2}{\hubble^3} \left[1 - 2 \left(1 + \decel \right) z + \frac{5}{12} \left(7 + 14 \decel - 2 \jerk + 9 \decel^2 \right) z^2 \right] \nonumber
\end{equation}
\smf{Can't find a reference for this equation\ldots}
}
%\begin{align}
%\frac{{\rm d}V}{{\rm d}z} \simeq & 4\pi \frac{c^3 z^2}{\hubble^3} \\
%& \left[1 - 2 \left(1 + \decel \right) z + \frac{5}{12} \left(7 + 14 \decel - 2 \jerk + 9 \decel^2 \right) z^2 \right] \nonumber
%\end{align}
The upper limit to the integral, $\zmax$, must be chosen such that there is negligible probability of a merger at $\zmax$ being detected: we find that $\zmax=0.44$ suffices for our setting. For our fiducial parameter set, $\lambda = 25160$; the particular Poisson draw we obtain yields a total of 25241 mergers.

For each merger, we draw a cosmological redshift assuming a constant source-frame rate (see, e.g., Eq. 28 of Ref.~\cite{Mortlock_etal:2019}), along with an isotropically distributed angular sky position, inclination angle, phase and polarization angle. We draw uniformly distributed BH masses from $\prob(\mbh) = \uniform(2.5\msun, 40\msun)$, taking the upper limit from low-metallicity\footnote{Using solar metallicity simulations would reduce this upper limit to $\sim12\,\msun$, likely increasing, as we will see, the number of EM-detectable NSBH mergers.}
%($Z = 0.0002$)
binary population synthesis simulations~\cite{Kruckow_etal:2018} and extending to low masses to reflect the detection of objects in the purported mass gap~\cite{LVC:2020O3acat}. NS masses are drawn from $\prob(\mns) = \uniform(1\msun, 2.42\msun)$, with an upper limit chosen to match that of the DD2 EOS \smf{Samaya, what's the best citation here? There seem to be many different citations for this EOS in the literature\ldots}, and BH and NS spin magnitudes from the uniform distributions $\prob(\abh) = \uniform(0, 0.99)$ and  $\prob(\ans) = \uniform(0, 0.05)$, assuming they are oriented isotropically. Following Ref.~\cite{Foucart_etal:2018}, we use the component masses, NS radii and BH spins to calculate the baryonic mass ejected by each merger. Note that this formula 1) requires the assumption of a NS EOS (we use DD2) and 2) has been calibrated using simulations without precession. We use the same EOS to calculate tidal deformabilities for the NSs, setting the BH deformabilities to zero. Finally, we generate a peculiar velocity $v$ for each merger from a zero-mean Gaussian with $500\,\kms$ standard deviation.

With the ground truth parameters in hand, we generate the data for each merger and apply our selection criteria.  To determine the impact of different physical effects on our results, we simulate two populations using different waveform approximants: the BNS-focused \imrpfull~\cite{Dietrich_etal:2019} and NSBH-specific \seobnrfull~\cite{Matas_etal:2020} (hereafter \imrp\ and \seobnr). \smf{Samaya, could you please describe the differences here?} As the \seobnr\ waveform requires aligned or anti-aligned spins, we zero the transverse NS and BH spins after sampling them isotropically (mimicking, in some sense, spins becoming aligned over time). For each merger, we generate a 32-second segment of noisy\footnote{Using spectra from \url{https://dcc.ligo.org/LIGO-T2000012/public}.} GW data $\dgw$ per waveform using the same random seed and a frequency range of 20-2048 Hz, considering it detected if the network signal-to-noise ratio is at least $\snrmin = 12$. We assume that the GW detectors operate in concert with an EM followup program capable of detecting all mergers with ejecta mass greater than $\mejmin = 0.01\,M_\odot$, modeling for the first time \smf{is this strictly true given the Chen paper (2006.02779)? } a hybrid GW-EM selection function. Finally, we generate noisy redshift and peculiar velocity estimates by drawing from $\prob(\zobs|z)=\normal(z,0.001)$ and $\prob(\vobs|v)=\normal(v,200\,\kms)$, respectively. Of the 25241 simulated mergers, 2477 (2954) are detected in GWs using the \imrp\ (\seobnr) waveform, 99 (75) of which have sufficient ejecta to be detected in EM; 62 appear in both samples. The SNRs for \seobnr\ waveforms are, on average, 5.9\% larger than their \imrp\ counterparts \smf{Samaya, do you know if there's an obvious published reason this is true?}, resulting in the GW-detected \seobnr\ sample containing $\sim$500 more objects. Zeroing the transverse spins for use with the \seobnr\ waveform, however, has the side-effect of reducing the typical ejecta mass~\cite{Foucart_etal:2018} and hence the final GW+EM-detected sample.

\begin{figure*}[ht!]
\includegraphics[width=18cm]{{nsbh_pop_H1+_L1+_V1+_K1+_A1_d_32.0_mf_20.0_rf_14.0_dndz_rr_ubhmp_2.5_40.0_unsmp_1.0_2.4_bbhsp_h_0_constraints_binned_by_par}.pdf}
\caption{Distributions of a subset of parameters from our \seobnr\ (top) and \imrp\ (bottom) samples, as drawn from the prior (dotted), selected by GW SNR (dashed) and selected by GW and EM emission (colored histograms). The bins are colored by the fractional \hubble\ uncertainty the mergers within the bin achieve: the yellowest bins are most informative. \label{fig:pops}}
\end{figure*}

To illustrate the impact of our selection function, we plot histograms of our full population (dotted lines), GW-selected events (dashed lines) and GW+EM-selected mergers (colored bars) for a subset of our parameters in Figure~\ref{fig:pops}. The prior curves are identical in both cases apart from the BH spin magnitude column (far right), where zeroing the transverse spins has made the \seobnr\ population's distribution non-uniform. The primary impact of the GW SNR threshold is, as expected, to select nearby mergers; it also imparts {\it very slight} preferences for low mass ratios and prograde BH $z$ spins \smf{Samaya, I feel like I've read this somewhere. Do Foucart/V\&C mention this?}. It is interesting to note that the GW-selected \seobnr\ distance distribution is broader than that of \imrp\ and peaked at slightly higher distances: this is a direct consequence of the \seobnr\ injections' systematically higher SNRs.

The ejecta-mass threshold (i.e., EM selection) strongly impacts the observed distributions. The GW+EM-detected mergers are shifted to even smaller distances, particularly for the \seobnr\ waveform, as the low-mass-ratio systems which produce significant ejecta mass can only be detected nearby. There is a very strong preference for mass ratios under 10 (again, especially so for \seobnr)\footnote{Populations with near-solar metallicity produce NSBH systems with lower BH masses~\cite{Kruckow_etal:2018} and hence a larger population of GW+EM-detectable mergers.} and large spins \smf{Samaya, can you add appropriate citations please?}, and the preference for positive $z$ spins is much more pronounced. The differences between the two waveforms' GW+EM-selected distributions are slightly obfuscated by the small sample sizes, but the \seobnr\ sample is shifted towards lower distances and mass ratios. As the \seobnr\ mergers' BH spin magnitudes are smaller than their \imrp\ counterparts', they require smaller mass ratios to produce significant ejecta~\cite{Foucart_etal:2018}, and the resulting systems are harder to detect at distance. Finally, we note that our EM selection function is a first-pass approximation. Our implementation captures our current best understanding of the dependence on ejecta mass, but does not incorporate any viewing-angle dependence \smf{all: would this be expected to reinforce the GW selection?} or instrumental effects \smf{What citations? Christian's BNS paper for starters, also 2011.08863? 1807.05226? 2006.02779}. Improving the physical models used to model EM selection is ongoing work in the field.

% % % % % % % % % % % % % % % % % % % % % % % % % % % % % %

\section{Methods} \label{sec:methods}

The probabilistic inference of the Hubble constant from catalogues of compact object mergers has been described in detail in the literature~\cite{Schutz:1986,Dalal:2006,Nissanke_etal:2010,Taylor_etal:2012,Nissanke_etal:2013,Abbott_etal:2017a,Chen_etal:2018,Fishbach_etal:2018,Feeney_etal:2018,Mandel_etal:2018,Gray_etal:2019,Mortlock_etal:2019,Vitale_etal:2020}. In the following, we adopt a slight variant of the formalism set out in Ref.~\cite{Mortlock_etal:2019}, whose Figure 9 depicts a network diagram for the model we use to describe the data\footnote{The only addition required to the network diagram of~\cite{Mortlock_etal:2019} is the dependence of the selection, $S$, on an intrinsic parameter: the merger's ejecta mass.}. The posterior we evaluate is given by
\begin{widetext}
\begin{align}
\prob \left( \hubble, \decel, \Gamma, \{z, v\} | N, \left\{ \dgw, \hat{z}, \hat{v} \right\}, \snrmin, \mejmin \right) & \propto
\prob \left( \hubble, \decel, \Gamma \right) \exp \left( -\bar{N} \left[ \hubble, \decel, \Gamma, \snrmin, \mejmin \right] \right) \\
& \times \prod_{i = 1}^{N} \frac{\Gamma}{1 + z_i} \frac{{\rm d}V}{{\rm d}z} \left[ \hubble, \decel \right] \prob (\dgw_i | d \left[ z_i, \hubble, \decel \right]) \prob(v_i) \prob(\hat{v}_i | v_i) \prob(\hat{z}_i | z_i), \nonumber
\end{align}
\end{widetext}
%
% one wide line, dropping dependencies
%\begin{widetext}
%\begin{align}
%{\rm P} \left( \hubble, \decel, \Gamma, \{z, v\} | N, \left\{ \hat{\bm{x}}, \hat{z}, \hat{v} \right\}, \snrmin, \mejmin \right) & \propto
%{\rm P} \left( \hubble, \decel, \Gamma \right) e^{-\bar{N}} \prod_{i = 1}^{N} \frac{\Gamma}{1 + z_i} \frac{{\rm d}V}{{\rm d}z} {\rm P} \left( \hat{\bm{x}}_i | d \right) P(v_i) P(\hat{v}_i | v_i) P(\hat{z}_i | z_i),
%\end{align}
%\end{widetext}
%
% dropped dependencies, trying to squeeze into single column
%\begin{align}
%{\rm P} & \left( \hubble, \decel, \Gamma, \{z, v\} | N, \left\{ \hat{\bm{x}}, \hat{z}, \hat{v} \right\}, \snrmin, \mejmin \right) \propto \\
%& {\rm P} \left( \hubble, \decel, \Gamma \right) e^{-\bar{N}} \prod_{i = 1}^{N} \frac{\Gamma}{1 + z_i} \frac{{\rm d}V}{{\rm d}z} {\rm P} \left( \hat{\bm{x}}_i | d \right) P(v_i) P(\hat{v}_i | v_i) P(\hat{z}_i | z_i), \nonumber
%\end{align}
where $N$ and $\nexp$ are the actual and expected number of mergers detected, respectively, curly brackets denote sets of quantities and bold denotes per-merger vectors.

We infer the parameters of this model in two parts, using two sampling methods. First, we first process each merger individually in order to obtain the GW likelihoods marginalized over all parameters $\boldsymbol{\theta}_i$ other than the luminosity distance $d$ to the merger;
%\begin{equation*}
%\prob(\dgw_i | d[ z_i, \hubble, \decel ]) = \int {\rm d}\boldsymbol{\theta}_i \prob(\boldsymbol{\theta}_i) \prob(\dgw_i | d[ z_i, \hubble, \decel ], \boldsymbol{\theta}_i),
%%\label{eq:marge_like}
%\end{equation*}
%where
%\begin{equation*}
%d \simeq \frac{cz}{\hubble} \left[1 + \frac{1}{2} \left(1 - \decel \right) z + \frac{1}{6} \left(-1 + \decel - \jerk + 3 \decel^2 \right) z^2 \right].
%%\label{eq:distance}
%\end{equation*}
$\boldsymbol{\theta}_i$ here comprising the $i^{\rm th}$ merger's component masses, spin magnitudes and orientations (where appropriate), inclination, polarization angle, NS tidal deformability, and time and phase at coalescence. We adopt priors identical to the distributions used in the generative model for all parameters other than the masses. Convergence is greatly improved by sampling chirp masses and mass ratios instead of component masses, and we therefore sample using interim priors that are uniform in these parameters (over the ranges permitted by our component-mass extrema), before importance-sampling the outputs to reinstate our desired component-mass priors. The marginal GW likelihoods are sampled with the \texttt{pypolychord} nested sampler~\cite{Handley_etal:2015a,Handley_etal:2015b}, wrapped by \texttt{bilby}~\cite{Ashton_etal:2019}, using 1000 live points and \texttt{bilby}'s \texttt{marginalize\_phase}, \texttt{\_time} and \texttt{\_distance} settings. Each 15/11-dimensional \imrp\ (\seobnr) sampling run takes 6-14 (4-6) days to complete on one Intel Xeon 2.70GHz CPU.

Given the marginal GW likelihoods, we use No-U-Turn Sampling as implemented by the \texttt{pystan} package~\cite{pystan} to infer the cosmological and population parameters. To connect to the cosmological parameters, we adopt a third-order distance-redshift relation matching our volume element~\cite{Visser:2004}, varying \hubble\ and \decel\ but fixing the jerk to one. We assume a broad Gaussian prior on \hubble, $\prob(\hubble)=\normal(70,20)$ \kmsmpc, a truncated Gaussian prior on \decel, $\prob(\decel)=\step(\decel+2)\step(1-\decel)\normal(-0.5,0.5)$, and a log-uniform prior on the rate, $\prob(\rate) \propto 1/\rate$. To use \texttt{pystan}, we must be able to sample all parameters from analytic distributions. We therefore perform a Gaussian Mixture Model fit to each merger's marginal distance likelihood using \texttt{pomegranate}~\cite{Schreiber:2017}. We fit each likelihood with an integer grid of 2-10 mixture components, repeating 10 times at each grid point and selecting the best fit using the Akaike Information Criterion~\cite{Akaike:1974}. Finally, we must evaluate the expected number of detected mergers $\nexp$ at each sampled value of the cosmological and population parameters. We do so by re-simulating the catalogues 100 times at each point of a 5x5 grid in $\{H_0,q_0\}$ assuming our fiducial rate, and interpolating the results using a 2D fourth-order interpolant. The dependence on the sampled rate is captured by multiplying the interpolation coefficients by $\Gamma/\Gamma_{\rm fid}$. The resulting 153/193-dimensional \texttt{pystan} inference runs take less than a minute to generate 20,000 well converged samples on a 3.1 GHz Intel Core i7 CPU. The set of true redshifts and peculiar velocities are uninteresting for the purposes of cosmology inference, and we therefore marginalize over these parameters when quoting the results below.

% % % % % % % % % % % % % % % % % % % % % % % % % % % % % %

\section{Results} \label{sec:results}

Processing the simulated \seobnr\ and \imrp\ catalogues through our two-stage inference pipeline produces the cosmology and population parameter posteriors shown in Fig.~\ref{fig:cosmo}. In both cases, the recovered \hubble, \decel\ and rate posteriors are completely consistent with the ground truth values, indicating, as expected, that the selection effects are correctly accounted for~\cite{Mortlock_etal:2019}. The 68\% credible intervals on the near-Gaussian $\hubble$ marginal posteriors are $68.8 \pm 1.6\,\kmsmpc$ for the \seobnr\ sample and $66.6 \pm 1.0\,\kmsmpc$ for \imrp. Recall that the \imrp\ sample contains 99 objects to the \seobnr\ sample's 75, and we should therefore expect that the \imrp\ sample's $\hubble$ posterior be roughly 13\% narrower than that of the \seobnr\ sample. The remaining reduction, therefore, reflects the ability for precessing spins to break the distance-inclination degeneracy \smf{Samaya, what are the best citations here?}. This additional constraining power is equivalent to an approximate doubling of the catalogue size.

The \hubble\ uncertainties we find for both waveforms are comparable to the current Cepheid-SN distance ladder precision~\cite{Riess_etal:2019}. NSBH populations -- should they produce EM counterparts and occur at rates roughly matching our assumptions -- will therefore strongly inform the outcome of the current \hubble\ tension, particularly when combined with accompanying BNS populations, likely of comparable size~\cite{Chen_etal:2018,Feeney_etal:2018,Vitale_Chen:2018,Mortlock_etal:2019}. The mergers are also informative about the deceleration parameter, $q_0$, shrinking its uncertainty from 0.5 to 0.32 or 0.27, depending on the waveform \smf{cf Shafieloo paper (1812.07775)?}. This further implies that NSBH catalogues will be able to begin constraining parameters such as the matter density and dark energy EOS (in the context of \lcdm\ and extended models), complementary to BBH results from higher redshifts~\cite{Farr_etal:2019,Chen_etal:2020,Mukherjee_etal:2020}. The merger rates are recovered with roughly 10\% precision. \smf{Samaya, should we compare to anything here? Could you please add relevant citations thank you!}

\begin{figure*}[ht!]
\includegraphics[width=8cm]{{pc_nsbh_pop_H1+_L1+_V1+_K1+_A1_d_32.0_mf_20.0_rf_14.0_dndz_rr_ubhmp_2.5_40.0_unsmp_1.0_2.4_bbhsp_seobnr_aligned_gmm_fits_rate_cosmo_post_triangle_plot}.pdf}\includegraphics[width=8cm]{{pc_nsbh_pop_H1+_L1+_V1+_K1+_A1_d_32.0_mf_20.0_rf_14.0_dndz_rr_ubhmp_2.5_40.0_unsmp_1.0_2.4_bbhsp_gmm_fits_rate_cosmo_post_triangle_plot}.pdf}
\caption{Cosmological and population parameter posteriors inferred for the simulated \seobnr\ (left) and \imrp\ (right) NSBH samples.\label{fig:cosmo}}
\end{figure*}

To obtain a picture of the parameter combinations that are most important for the \hubble\ constraints, we return to Figure~\ref{fig:pops}. Here, the colors of the histogram bins indicate the fractional \hubble\ uncertainty the mergers within each bin attain, with the most-constraining bins colored yellow and the least-constraining blue. For both waveforms, the bulk of the \hubble\ constraining power comes from mergers with distances less than 700 Mpc, not just the nearest, loudest events. For the \imrp\ mergers, all mass ratio bins less than $\sim$8$\msun$ contribute equally, despite the frequency dropping rapidly with mass ratio. For the \seobnr\ mergers, the constraints are instead driven by the lowest-mass ratio events \smf{Samaya, are large masses more important with precessing spins?}. From the \imrp\ spin panels, it is clear that that highest-spin events constrain \hubble\ most strongly, with the full \hubble\ constraint coming almost entirely from the highest-spin (and most populated) bin. The \seobnr\ constraint, on the other hand, is sourced by events with a broader range of spins, though this is likely driven by the prior.

We further highlight the importance of precession in breaking the distance-inclination degeneracy in Figure~\ref{fig:waveforms}. In the first three panels we plot distance and inclination constraints for a selection of mergers when using the \seobnr\ (red, filled) and \imrp\ (grey, filled) waveforms.
% {\bf [BH cartesian spin vector (-0.20,-0.24, 0.92); mag 0.97, tilt 0.33]}
For the first two mergers, detected with high SNR, the long degeneracies present in their \seobnr\ posteriors are almost completely broken when using the \imrp\ waveform, in which the spins precess. We illustrate this point further by re-running the \imrp\ case assuming aligned spins, having set the transverse spins to zero. These results are overlaid as dashed dark red contours. Both distance-inclination degeneracies blow up, increasing the distance uncertainties by a factor of over three, with commensurate consequences for the mergers' ability to constrain $H_0$. In the third panel, we show equivalent posteriors for the \imrp\ merger whose BH spin is closest to being aligned: the effect of switching waveforms here is markedly reduced. The impact on the population level is clear from the right-hand panel of Figure~\ref{fig:waveforms}, in which we plot the distributions of individual mergers' fractional errors on distance (dashed) and \hubble\ (solid) when using the \seobnr\ (red) and \imrp\ (grey) waveforms. The \seobnr\ distributions are shifted to significantly higher errors than their \imrp\ counterparts, despite the \seobnr\ mergers typically having higher SNRs. The smallest percentage error for any individual merger is 2.8\% for the \imrp\ case and 6.1\% for \seobnr; the medians are 13.2\% and 17.3\%, respectively. The \hubble\ constraints imparted by both ``golden'' and bulk events are therefore stronger when spins precess significantly. Finally, we note from the dashed curves that peculiar velocity {\it and redshift} uncertainties strongly suppress the constraining power of the nearest and loudest events.

\begin{figure*}[ht!]
\includegraphics[width=18cm]{{nsbh_pop_H1+_L1+_V1+_K1+_A1_d_32.0_mf_20.0_rf_14.0_dndz_rr_ubhmp_2.5_40.0_unsmp_1.0_2.4_bbhsp_frac_errs}.pdf}
\caption{Left and center: distance and inclination posteriors for a selection of mergers, simulated and sampled using the \imrp\ waveform with precessing (grey filled) and aligned (dark red dashed) spins, and using \seobnr\ with aligned spins (red). The selection includes the highest-SNR merger common to both catalogues (left) and the \imrp\ merger whose BH spin is closest to being aligned (second from right). Right: distributions of fractional uncertainties on luminosity distance (dotted) and \hubble\ (solid) from individual mergers from our \imrp\ (grey) and \seobnr\ (red) NSBH catalogues. \label{fig:waveforms}}
\end{figure*}

% % % % % % % % % % % % % % % % % % % % % % % % % % % % % %

\section{Conclusions} \label{sec:conclusions}
Full simulation inc noise. Combined GW+EM selection. Unbiased, with excellent precision. Particular physics / waveform matters in detail: precession buys you factor of two in sample size. Will inform \hubble\ debate either way.

% % % % % % % % % % % % % % % % % % % % % % % % % % % % % %

\begin{acknowledgments}
\smf{All: please add your acknowledgments, and please fill out your contributions as defined in the link below.} S.M. Feeney is supported by the Royal Society.
{\bf Author contributions.} S. M. Feeney: Conceptualization, Methodology, Software, Investigation, Visualization, Writing (Original draft preparation). H. V. Peiris: . S. M. Nissanke. D. J. Mortlock: . \url{https://www.elsevier.com/authors/journal-authors/policies-and-ethics/credit-author-statement}
\end{acknowledgments}

% % % % % % % % % % % % % % % % % % % % % % % % % % % % % %

\bibliography{references}% Produces the bibliography via BibTeX.

\end{document}
%
% ****** End of file apssamp.tex ******